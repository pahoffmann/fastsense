\documentclass{beamer}
\usepackage[utf8]{inputenc}
\usepackage[T1]{fontenc}
\usepackage{lmodern}
\usepackage{tikz}
\usepackage[absolute,overlay]{textpos}

\title{Meilenstein 3}
\author{Projektgruppe FastSense}
\date{25. Januar 2021}

\usecolortheme{seahorse}
\definecolor{dark}{rgb}{0, 0.1, 0.3}
\definecolor{light}{rgb}{0.9, 0.933, 1}
\setbeamercolor{normal text}{fg=black}
\setbeamercolor{structure}{fg=dark}
\setbeamercolor{footline}{fg=black}
\setbeamercolor{frametitle}{fg=light,bg=dark}
\setbeamertemplate{itemize items}[circle]
\beamertemplatenavigationsymbolsempty
\addtobeamertemplate{navigation symbols}{}{
    \usebeamerfont{footline}
    \usebeamercolor[fg]{footline}
    {\footnotesize \insertframenumber\\\vspace{0.15cm}}
}
\setbeamertemplate{title page}{
\insertauthor\\\vspace{0.5cm}
\begin{LARGE}\textbf{\inserttitle}\end{LARGE}\\\vspace{0.5cm}
\insertdate
}

\begin{document}

{\setbeamertemplate{navigation symbols}{}
\begin{frame}
\titlepage
\end{frame}}

\begin{frame}{Inhalt}
\tableofcontents
\begin{textblock*}{1cm}(0cm,3.2cm)
\begin{tikzpicture}
\draw [white] (0, 2.35) rectangle +(1, 1); % Hier den y-Wert anpassen, falls sich das Inhaltsverzeichnis noch ändert
\draw [light, ultra thick] (1, 0) -- (7, 0);
\node [right, dark] at (7, 0) {Live Demonstration};
\end{tikzpicture}
\end{textblock*}
\end{frame}

\section{Recap MS\,2}
\begin{frame}{\secname}
TODO
\end{frame}

\section{Motivation MS\,3}
\begin{frame}{\secname}
TODO
\end{frame}

\section{Was haben wir wirklich gemacht?}
\begin{frame}{\secname}
TODO
\end{frame}

\subsection{Drohne, Laserscanner}
\begin{frame}{\subsecname}
TODO
\end{frame}

\subsection{Aufbau}
\begin{frame}{\subsecname}
TODO
\end{frame}

\subsection{Base Design}
\begin{frame}{\subsecname}
TODO
\end{frame}

\subsection{Kommunikation}
\begin{frame}{\subsecname}
TODO
\end{frame}

\subsection{Algorithmus}

\subsubsection*{Preprocessing}
\begin{frame}{\subsecname: \subsubsecname}
TODO
\end{frame}

\subsubsection*{Registrierung}
\begin{frame}{\subsecname: \subsubsecname}
TODO
\end{frame}

\subsubsection*{Asynchronität}
\begin{frame}{\subsecname: \subsubsecname}
TODO
\end{frame}

\subsubsection*{Memports}
\begin{frame}{\subsecname: \subsubsecname}
TODO
\end{frame}

\subsection{Mesh Rekonstruktion}
\begin{frame}{\subsecname}
TODO
\end{frame}

\subsection{Paper}
\begin{frame}{\subsecname}
TODO
\end{frame}

\section{Evaluation}

\subsection{Zeit}
\begin{frame}{\secname: \subsecname}
TODO
\end{frame}

\subsection{Power Consumption}
\begin{frame}{\secname: \subsecname}
TODO
\end{frame}

\subsection{Qualität}
\begin{frame}{\secname: \subsecname}
TODO
\end{frame}

\section{Ausblick}
\begin{frame}{\secname}
\begin{itemize}
\item{FastSense Paper}
\item{Loop Closing}
\item{Drohne}
\item{Modulares Design}
\item{Grillen bei Mario}
\end{itemize}
\end{frame}

\begin{frame}{Ende}
\centering\LARGE
Vielen Dank für Ihre Aufmerksamkeit!\\\vspace{1cm}
Fragen?
\end{frame}

\end{document}
